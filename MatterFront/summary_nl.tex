\chapter{Samenvatting}
Er is een tendens gaande om de huidige spoorinfrastructuur te digitaliseren omdat er verwacht wordt dat het spoortransport de komende jaren een belangrijke factor zal zijn in het terugdringen van klimaatverandering. Vergeleken met andere middelen van transport zoals lucht- en wegentransport word spoortransport gezien als een milieuvriendelijker alternatief. Met de verwachte toename van zowel goederen- en personenvervoer via het spoor, zal de druk op de bestaande spoorinfrastructuur toenemen. Om de betrouwbaarheid, beschikbaarheid, onderhoudbaarheid, en veiligheid te borgen van de spoorinfrastructuur op een effici\"{e}nte manier, is het van belang om een nauwkeurig en actueel digitaal model te hebben van de huidige staat. Deze modellen kunnen voor diverse taken worden gebruikt zoals planningen, het volgen van de onderhoudsstaat, automatische inventarisaties, autonome werkzaamheden en uiteindelijk voorspelbaar onderhoud.

Het geautomatiseerd verkrijgen van deze modellen is uitdagend omdat het spoornetwerk erg uitgebreid is en uit vele componenten bestaat. Aan de ene kant bevat de spooromgeving discrete componenten zoals seinen, bovenleidingsportalen, en relaiskasten. Aan de andere kant bevat de spooromgeving continue objecten zoals het spoor zelf en de rijdraden. Wat eveneens bijdraagt aan de complexiteit is het feit dat de hoeveelheid onderscheidbare objecten rond het spoor enorm is. Vooral in landen met een rijke spoorhistorie zoals Nederland. Deze factoren, gecombineerd met de grote diversiteit binnen een objectcategorie, maken het digitaliseren van de spooromgeving tot een uitdagende taak.

Om de digitaliseringsslag te verwezenlijken, wordt puntenwolkdata gezien als een veelbelovend formaat. Puntenwolken worden opgenomen met behulp van Light Detection and Ranging (LiDAR) technologie. Deze technologie maakt gebruik van laserlicht om de omgeving te scannen. Dit wordt gedaan door een puls van onzichtbaar laserlicht uit te zenden en de tijd te meten totdat de reflectie terugkomt. Deze tijd is direct evenredig met de afstand tot het object. Deze afstand, gecombineerd met de bekende ori\"{e}ntatie van de laser, maken het mogelijk om de 3D positie van de reflectie te berekenen relatief tot de sensor. Puntenwolken maken het mogelijk om objecten in 3D met hoge nauwkeurigheid in te meten, onafhankelijk van externe lichtingsomstandigheden. 

Dit proefschrift richt zich op het digitaliseren van de spooromgeving met behulp van puntenwolkdata. Het proefschrift volgt een benadering van een hoog naar laag abstractieniveau. Eerst worden objecten gedetecteerd in de puntenwolk op grove schaal, daarna worden deze objecten verder ontleed in kleinere betekenisvolle componenten. Deze twee stappen omvatten de vertaling van de fysieke wereld naar de cyberwereld. Om de cyclus te sluiten, en weer van de cyberwereld naar de fysieke wereld te keren, wordt er in dit proefschrift ook gekeken naar de intu\"{i}tieve visualisatie van de informatie uit de cyberwereld met behulp van augmented reality (AR).

Het proefschrift begint met hoe objecten op grote schaal vanuit puntenwolkdata kunnen worden gedetecteerd. Dit wordt gedaan door bestaande modellen uit het domein van zelfrijdende auto's te evalueren op toepasbaarheid in het spoorwezen. Een belangrijk inzicht uit deze exercitie is dat de locatienauwkeurigheid van deze modellen nog niet voldoende is voor het cre\"{e}ren van nauwkeurige digitale modellen van het spoor. Een ander inzicht is dat de modellen nog niet generiek genoeg zijn zodat ze eenvoudig op nieuwe stukken spoor kunnen worden toegepast. Er wordt aangetoond dat transfer learning een oplossing kan zijn voor dit probleem.

Om de grotere objecten verder te ontleden, en zo het detailniveau te verhogen van het digitale model, wordt er in dit proefschrift gekeken naar semantische segmentatie. Het doel van semantische segmentatie is om op puntniveau een categorie toe te kennen voor alle punten in de puntenwolk. Deze decompositie kan dan worden gebruikt om \textsc{CAD} modellen uit een \textsc{CAD} bibliotheek op te halen en deze in het digitale model te plaatsen. Dit proefschrift gaat in op de semantische segmentatie van bovenleidingsportalen die zijn ingewonnen met hele hoge resolutie puntenwolkdata. De bovenleidingsportalen worden ontleed in veertien verschillende klassen, een algehele mIoU score van 71\% is behaald voor deze taak.

Objectdetectie en semantische segmentatie omvatten de vertaalslag van de fysieke wereld naar de cyberwereld. Om de cyclus weer te sluiten, en van de cyberwereld naar de fysieke wereld terug te keren, wordt er gekeken naar een intu\"{i}tieve visualisatie van deze informatie. Dit wordt gedaan door gebruik te maken van een speciale augmented reality bril. Met behulp van deze technologie is er een concept gecre\"{e}erd waarbij de gebruiker in staat is om met handgebaren en spraak interacties aan te gaan met een visualisatie van een puntenwolk. Voor de gebruiker lijkt het of de puntenwolk een fysiek tastbaar object is.

Samenvattend, heeft dit proefschrift onderzocht hoe de spooromgeving gedigitaliseerd kan worden aan de hand van puntenwolkdata. Het proefschrift bevat waardevolle inzichten en aanknopingspunten voor vervolgonderzoek om dit relatief nieuwe onderzoeksveld tot een hoger niveau te tillen. Een van de veelbelovende aanbevelingen is om naar self-supervised learning (SSL) te kijken. Met deze techniek is het mogelijk om met data zonder labels al kennis in een model te vangen. Met relatief weinig labels is het dan toch mogelijk om hoge nauwkeurigheden te halen. Dit verlicht de vaak tijdrovende en dure taak om veel gelabelde data te verzamelen.