\section{Conclusion}\label{sec:stoa:conclusion}
This chapter has reviewed the literature on using point clouds in the context of railway infrastructure. We have divided the literature into pre-processing, modelling, and digital twin creation. We have described different techniques describing their strengths and weaknesses.
We have focused on literature concerning railway infrastructure and point clouds, thus excluding literature studying the presence of foreign objects in railway infrastructure that could be an exciting topic concerning predictive maintenance.

The current trend for modelling is focused on machine learning-based techniques, particularly deep learning-based techniques. However, contrary to the usual practice in artificial intelligence research, data and implementation code is not published for most research, hindering reproducibility and cross-comparison. We emphasise a need towards open publication of data and implementation for scientific research, enabling breakthroughs in this area of research. 

Besides the typical challenges associated with point cloud data, railway data have additional challenges, such as variation in object types and sizes, vast sizes of data, and the critical nature of infrastructure hindering the open publication of point cloud data.

As an alternative future research direction, we propose to focus on hybrid methods to which combine the strengths of structure-based and machine learning-based techniques. Also we propose a focus on developing large pre-trained models for point clouds, which, in general, will enable transfer learning that reduces the training efforts. Some preliminary results exploring this route are presented in Chapter~\ref{chap:objdet}. To alleviate the burden of labelling data, machine learning techniques focused on partially labelled data can help and possibly even improve the state of the art. From the perspective of digital twins, in the context of railway infrastructure and point cloud, there are still many open research opportunities.

To conclude, the current state of the art is varied in terms of techniques and technologies and could be further strengthened with the use of hybrid methods, multimodal-model and ensemble learning approaches, partially labelled data approaches, and the creation of digital twin to reap full-scale benefits of predictive maintenance enabled and digitalised railways.