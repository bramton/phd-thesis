\section{Introduction}\label{sec:stoa:introduction}
The conversion of point clouds into a digital twin is a complex, multi-step process. An initial step often involves segmenting the cloud into different railway-related objects, such as tracks or poles. The interest in point cloud segmentation is not exclusive to railway infrastructure monitoring, it is rather crucial for other domains, especially in autonomous driving~\cite{li2020deep} or infrastructure monitoring~\cite{mirzaei20223d} among many other applications~\cite{nguyen_3d_2013}. 

LiDAR technology, while not new, has gained renewed interest due to the surge of machine learning-based techniques~\cite{li20193d}. The research on the crosscut between LiDAR and machine learning is multi-faceted, emphasising the need to collate and analyse literature on point clouds within the railway monitoring and predictive maintenance domain.

Previous systematic reviews have addressed 3D data collection and analysis, railway datasets, and point cloud analysis methods. For instance, \cite{corongiu2018data} discusses data integration of different domains to obtain a 3D dataset of the railway environment. \citeauthor{dong2020registration} reviews methods for the registration of terrestrial laser scanner point clouds~\cite{dong2020registration}. Different datasets of the railway environment are discussed in~\cite{pappaterra2021systematic}. Techniques for point cloud analysis are reviewed in~\cite{li20193d} and~\cite{klimkowska2022detailed}. However, there seems to be a gap in systematic reviews specifically targeting point cloud segmentation or object detection methods.

This review aims to provide an overview of the current state-of-the-art methods, models, and technologies that can be used to digitalise railway infrastructure for monitoring and maintenance. 
Railway scene, railway environment, and railway infrastructure are all closely related terms with similar meanings. To avoid ambiguity, we list the definitions below as used in this research:
 \begin{itemize}
     \item \emph{Railway scene:} All objects in the surroundings of the railway tracks including vegetation, urban buildings and foreign objects.
     \item \emph{Railway environment:} Synonym for railway scene. 
     \item \emph{Railway infrastructure:} All objects specifically belonging to the railway like tracks, poles, catenary arches, wires, etc. These are the objects of interest for this study.
 \end{itemize}

The remainder of this review is structured as follows: Section~\ref{sec:stoa:review-method} describes the review strategy. Section~\ref{sec:stoa:metaanalysis} provides metadata about the publications and includes a table summarising the characteristics of the datasets used in the included studies. The paper focuses on the gathering of literature for pre-processing (Section~\ref{sec:stoa:preprocessing}), modelling (Section~\ref{sec:stoa:modelling}), and the creation of a digital twin (Section~\ref{sec:stoa:digitaltwin}). The discussion section (Section~\ref{sec:stoa:discussion}) reflects on the gathered literature, identifies the literature gap, and provides future directions. Section~\ref{sec:stoa:conclusion} concludes the review.

\subsection{Author contribution}
This section highlights the author's contribution to the article~\cite{dekker23} on which this chapter is based. The contributions will be described using the comprehensive Contributor Role Taxonomy (CRediT), a taxonomy which has been widely adopted by a large number of journals~\cite{allen2019credit}. The related taxonomy terms are italicised for convenience. During the initialisation of the review article, I was involved in the \textit{data curation} task. This task involved screening the large corpus of manuscripts which had been extracted from various sources. As part of the \textit{writing -- original draft} phase, the sections about pre-processing (Section~\ref{sec:stoa:preprocessing}) and the available commercial software (Section~\ref{subsec:stoa:commsoft}) are contributed by me. Furthermore, the \textit{conceptualisation} and large parts of the contents of the research roadmap (Section~\ref{subsec:stoa:roadmap}) have been contributed by me. Regarding \textit{visualisation}, most of the tables and figures have been typeset by me. The \textit{conceptualisation} and implementation of using a Sankey diagram~\cite{schmidt2008sankey} (Figure~\ref{fig:stoa:prisma}) to \textit{visualise} how the final corpus of papers came into being was also by me. Finally, I have also been actively involved in the \textit{writing -- reviewing and editing} phase of the article.