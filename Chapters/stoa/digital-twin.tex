\section{Digital Twin}\label{sec:stoa:digitaltwin}
Digital twins can be defined as a bidirectional effortless data integration between a physical and a virtual machine in cyber space~\cite{fuller2020digital}. The terminology is defined extensively and is often confused with the notion of a static digital model, which is a mere virtual representation of the physical world~\cite{fuller2020digital}. However, as opposed to static models, digital twins are a dynamic digital representation of a physical object or system. This goes beyond just a mere 3D representation to include real-time data which allows the digital twin to simulate, predicate and optimise its physical counterpart its performance and operation. This of course requires a continuous flow of data from the physical object to maintain the accurate simulation and analysis. Throughout the lifetime of a digital twin, a real-time mirroring is required to reflect any changes or updates, which makes it a powerful tool in understanding, analysing and improving real-world objects and systems.

The use of digital twins is gaining traction in the context of railway infrastructure maintenance and monitoring. However, the scientific literature is scarce and the focus is on creation of digital model of railways with emphasis on 3D model generation and integration. 

In the context of digital twins, 3D model generation refers to the process of creating a 3D representation of the physical objects or environment in a digital twin environment. Incorporating 3D models enables realistic and immersive visualisation of railway assets allowing an assessment of the current state and prediction of the future state. 

The earliest work in the included studies is that of \citeauthor{benhmida2011from} which originates from 2011~\cite{benhmida2011from}. They have created VRML files with different coloured object classes using an ontological approach. \citeauthor{ariyachandra2020detection} have focused on generating Industry Foundation Classes (IFC) models. In their first paper, they fitted the detected point clusters into 3D models in IFC format~\cite{ariyachandra2020detection}. Their second paper was focused on generating dynamic IFC models of overhead line equipment configurations and merging them with point clusters using the iterative closest point algorithm~\cite{ariyachandra2020digital}. \citeauthor{soilan2019review} has also reported work on generating IFC files for track models and rail alignment~\cite{soilan2019review} (see also~\cite{soilan20213D,soilan2021fully}).

The other approaches were based on geometric-based modelling with curve fitting. The studies employing curve-fitting approaches include:  
\begin{itemize}
    \item Parameter estimation using Markov Chain Monte Carlo and curve fitting for rail track modelling~\cite{oudeelberink2013rail}.
    \item Rotational correction and projection of 3D points, fitting a pre-defined rail model, and interpolation using Fourier curve fitting~\cite{sahebdivani2020rail}.
    \item Piecewise straight line fitting for contact wire and dropper~\cite{xu2021vehicle-born}.
    \item Identifying and classifying railway cables (contact, catenary, return current)~\cite{chen2021railway}.
    \item Track model reconstruction using a third-degree polynomial function~\cite{jwa2015kalman}.
    \item Hybrid overlay technique using point data and polygons~\cite{kwoczynska2016elaboration}.
    \item Use of particle swarm optimisation to reconstruct the track considering the so-called track lining distance as an evaluation index~\cite{li2018methodology}.
    \item Heuristic-based point cloud pre-processing is used to segment railway to generate 3D model based on IFC requirements~\cite{soilan2021fully}. 
\end{itemize}
In the context of model generation \citeauthor{zhu2014the} have developed a complete building model by fusing facades with roofs, planar detection of buildings, ground model simplification, orthophoto for ground texture, and 3Ds Max for model visualisation~\cite{zhu2014the}.

Despite the scarcity of the literature in the application of digital twin technology in railway infrastructure, a wider adoption is expected to revolutionize railway monitoring and optimization. Digital twin has the potential of enhancing predictive maintenance enabling proactive issue resolution. This technology will not only optimize railway operations but also significantly improve its safety through simulations and hazard identification. 