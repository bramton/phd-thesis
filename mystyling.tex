%%%%%%%%%%%%%%%%%%%%%%%%%%%%%%%%%%%%%%%%%%%%%%%%%%%%%%%%%%%%%%%%%%%%%%%%%%%%%%%
%%%%%%%%%%%%%%%%%%%%%%% Miscalenous bits and pieces %%%%%%%%%%%%%%%%%%%%%%%%%%%
%%%%%%%%%%%%%%%%%%%%%%%%%%%%%%%%%%%%%%%%%%%%%%%%%%%%%%%%%%%%%%%%%%%%%%%%%%%%%%
\nouppercaseheads
\setsecnumdepth{subsection}
\nobibintoc
\setlength{\epigraphwidth}{0.6\textwidth}
 
% For a large list of chapter styles see the memoir manual.
\chapterstyle{ell}

% Reduce the space between top of page and chapter title
\setlength{\beforechapskip}{0pt}

% On a 'Chapter' page, put the page number on the right hand side.
\makeoddfoot{plain}{}{}{\thepage}

% Required when using the ctabular (continuous tabular) environment. 
\newfixedcaption{\freetabcaption}{table}

%%%%%%%%%%%%%%%%%%%%%%%%%%%%%%%%%%%%%%%%%%%%%%%%%%%%%%%%%%%%%%%%%%%%%%%%%%%%%%%
%%%%%%% Add thumbs with the chapter number on the edges of the pages %%%%%%%%%%
%%%%%%%%%%%%%%%%%%%%%%%%%%%%%%%%%%%%%%%%%%%%%%%%%%%%%%%%%%%%%%%%%%%%%%%%%%%%%%%
\newlength\thumbwidth
\setlength\thumbwidth{36pt}
\newlength\thumbheight
\setlength\thumbheight{24pt}
\newlength\thumbvspace
\setlength\thumbvspace{40pt}
\newlength\thumbradius
\setlength\thumbradius{4pt}

% Default style is called headings
% We need to add (\trimedge,\trimtop), because `current page' does not account for trims.
%     see: https://tex.stackexchange.com/questions/44416/is-memoirs-page-layout-incompatible-with-tikz
\makepagestyle{mainmatterstyle}
% The style for the recto (right) page
\makeoddhead{mainmatterstyle}{%
}%
{%
\begin{tikzpicture}[remember picture,overlay]
  \coordinate (top left) at ($(current page.north east) +(\trimedge,\trimtop) - (\thumbwidth,\uppermargin) - \value{chapter}*(0,\thumbheight+\thumbvspace) + (0,\thumbheight+\thumbvspace)$);
  \coordinate (bot right) at ($(top left) + (\thumbwidth+\thumbradius+\trimedge,0) - (0,\thumbheight)$);
  \fill[fill=sax-lgrey, rounded corners=\thumbradius] (top left) rectangle (bot right);
  \node at ($(bot right) + (0,0.5\thumbheight) - (0.5\thumbwidth+\thumbradius+\trimedge,0)$){\bfseries\color{white}\thechapter};
  %\node[circle,draw,fill=red] (c) at (0,0) {};
\end{tikzpicture}
}%
{%
\leftmark%
\makebox[0pt][l]{%
	\hspace{\marginparsep}
	{\color{sax-green}\rule{1pt}{1em}} % TODO: should rule be dropped a bit??
	\enskip\thepage	
}%
}%

% The style for the verso (left) page
\makeevenhead{mainmatterstyle}{%
\makebox[0pt][r]{%
	\thepage\enskip%
	{\color{sax-green}\rule{1pt}{1em}}
	\hspace{\marginparsep}
}%
\rightmark%
}%
{%
\begin{tikzpicture}[remember picture,overlay]
  \coordinate (top right) at ($(current page.north west) +(\trimedge,\trimtop) + (\thumbwidth,-\uppermargin) - \value{chapter}*(0,\thumbheight+\thumbvspace) + (0,\thumbheight+\thumbvspace)$);
  \coordinate (bot left) at ($(top right) - (\thumbwidth+\thumbradius+\trimedge,0) - (0,\thumbheight)$);
  \fill[fill=sax-lgrey, rounded corners=\thumbradius] (top right) rectangle (bot left);
  \node at ($(bot left) + (0,0.5\thumbheight) + (0.5\thumbwidth+\thumbradius+\trimedge,0)$){\bfseries\color{white}\fontsize{0.75\thumbheight}{0.75\thumbheight}\thechapter};
\end{tikzpicture}
}%
{}%


%%%%%%%%%%%%%%%%%%%%%%%%%%%%%%%%%%%%%%%%%%%%%%%%%%%%%%%%%%%%%%%%%%%%%%%%%%%%%%%
%%%%%%%%%% Define a 'blind' footnote, that is, without mark %%%%%%%%%%%%%%%%%%%
%%%%%%%%%%%%%%%%%%%%%%%%%%%%%%%%%%%%%%%%%%%%%%%%%%%%%%%%%%%%%%%%%%%%%%%%%%%%%%%
% Source: https://tex.stackexchange.com/questions/30720/footnote-without-a-marker
\newcommand\blfootnote[1]{%
  \begingroup
  %\setlength{\skip\footins}{0pt} % Does not work
  \renewcommand\thefootnote{}\footnote{#1}%
  \addtocounter{footnote}{-1}%
  \endgroup
}

%%%%%%%%%%%%%%%%%%%%%%%%%%%%%%%%%%%%%%%%%%%%%%%%%%%%%%%%%%%%%%%%%%%%%%%%%%%%%%%
%%%%%%%%%%%%%%%%%%% Change colour of the epigraph rule %%%%%%%%%%%%%%%%%%%%%%%%
%%%%%%%%%%%%%%%%%%%%%%%%%%%%%%%%%%%%%%%%%%%%%%%%%%%%%%%%%%%%%%%%%%%%%%%%%%%%%%%
% Source: https://tex.stackexchange.com/questions/251148/how-to-change-the-color-of-the-memoir-epigraph-rule
\newcommand{\epigraphrulecolor}{sax-green}
\makeatletter
\addtodef\@epirule{\color{\epigraphrulecolor}}{}
\makeatother

%%%%%%%%%%%%%%%%%%%%%%%%%%%%%%%%%%%%%%%%%%%%%%%%%%%%%%%%%%%%%%%%%%%%%%%%%%%%%%%
%%%%%%%%%%%%%%%%%%% Change colour of chapter heading lines. %%%%%%%%%%%%%%%%%%%
%%%%%%%%%%%%%%%%%%%%%%%%%%%%%%%%%%%%%%%%%%%%%%%%%%%%%%%%%%%%%%%%%%%%%%%%%%%%%%%
% WARNING: this also changes the colour of the rules in tables
% Copied from memoir.cls
\makeatletter
\renewcommand\chs@ell@helper[1]{%
  \par%
    \begin{adjustwidth}{}{-\chapindent}
      \arrayrulecolor{sax-green}% Custom addition
      \begin{tabularx}{\linewidth}{>{\raggedleft\arraybackslash}X|}%|emacs
        \leavevmode\chapnumfont #1\vphantom{1}%
        \hspace*{3.6pt}%
        \rule[-13.5pt]{0pt}{14.8mm}%
        \\%
        \hline%
      \end{tabularx}%
    \end{adjustwidth}%
    \par%
  }%
\makeatother